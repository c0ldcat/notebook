\documentclass[main.tex]{subfiles}
\begin{document}
\chapter{多元函数微分}

\section{多元函数的基本概念}
\subsection{平面点集}
\paragraph{邻域} $U(P_0, \delta) = \{ P \mid |P P_0| < \delta \}$
\subparagraph{去心邻域} $\mathring{U}(P_0, \delta) = \{ P \mid 0 < |P P_0| < \delta \}$
\subparagraph{领域描述点与点集关系}\figref{点与点集关系}
\begin{figure}[htp]
    \centering
    \psscalebox{0.75}{\begin{pspicture*}(-3,-5)(3,3)\pscircle[linewidth=2pt,linestyle=dashed,dash=1pt 1pt,fillcolor=black,fillstyle=solid,opacity=0.16](1.88466,1.50033){0.83489}\pscircle[linewidth=2pt,linestyle=dashed,dash=1pt 1pt,fillcolor=black,fillstyle=solid,opacity=0.16](0.32207,-0.55827){0.78091}\pscircle[linewidth=2pt,linestyle=dashed,dash=1pt 1pt,fillcolor=black,fillstyle=solid,opacity=0.16](2.27518,-2.21389){0.71610}\rput{60.73200}(-0.16,-0.25){\psellipse[linewidth=2pt](0,0)(2.86614,1.96518)}\begin{scriptsize}\psdots[dotstyle=*](1.88466,1.50033)\rput[bl](1.91967,1.58952){$P_3$}\psdots[dotstyle=*](0.32207,-0.55827)\rput[bl](0.36154,-0.46310){$P_1$}\psdots[dotstyle=*](2.27518,-2.21389)\rput[bl](2.31154,-2.12387){$P_2$}\end{scriptsize}\end{pspicture*}}
    \caption{点与点集关系}\label{点与点集关系}
\end{figure}
\begin{itemize}
    \item 内点: $P_1$
    \item 外点: $P_2$
    \item 边界点: $P_3$
    \item[-] 聚点: $\forall \delta > 0, \mathring{U}(P_0, \delta) \text{内有} E \text{中点}$
\end{itemize}
\paragraph{平面点集合分类}
\begin{itemize}
    \item 开集:
        \inlineps{\pscircle[linewidth=1pt,linestyle=dashed,dash=5pt 2pt,fillcolor=black,fillstyle=solid,opacity=0.33](0,0){1}}
    \item 闭集:
        \inlineps{\pscircle[linewidth=1pt,linestyle=solid,fillcolor=black,fillstyle=solid,opacity=0.33](0,0){1}}
    \item 连通集:
        \inlineps{\rput{90}(-0.43011,0.07747){\psellipse[linewidth=1pt,fillcolor=black,fillstyle=solid,opacity=0.27](0,0)(0.34095,0.22741)}\rput{65.44954}(0.51130,0.14720){\psellipse[linewidth=1pt,fillcolor=black,fillstyle=solid,opacity=0.23](0,0)(0.59639,0.39867)}}
        不是连通集
        \begin{itemize}
            \item 区域 (开区域): 连通开集
            \item 闭区域: 连通闭集
        \end{itemize}
    \item 有界集
    \item 无界集
\end{itemize}

\subsection{多元函数概念}
\paragraph{极限} $P_0(x_0, y_0) \text{是} D \text{的聚点}\ \exists A\ \forall \varepsilon\ P \in D \cap \mathring{U}(P_0, \delta)\ |f(P) - A| = |f(x,y) - A| < \delta$
\paragraph{连续} $P_0(x_0, y_0) \text{是} D \text{的聚点}\ \lim_{(x, y)\rightarrow(x_0, y_0)} f(x, y) = f(x_0, y_0)$
\begin{itemize}
    \item 间断点
\end{itemize}
\subparagraph{有界连续多元函数点性质}
\begin{itemize}
    \item 具有最大最小值
    \item 介值定理
    \item 一致连续性: 各个二维切面上都连续
\end{itemize}

\section{偏导数}
\paragraph{偏导数基础}
\[ \left. \frac{\partial f}{\partial x} \right|_{\begin{smallmatrix}x = x_0\\y = y_0\end{smallmatrix}} = f_x(x_0, y_0) \]
\subparagraph{计算方法} 把其他自变量看做常数
\paragraph{高阶偏导数}
    高阶混合偏导数中偏导数连续的条件下与求导次序无关
\subparagraph{拉普拉斯方程?} (P71)

\section{全微分}
\paragraph{偏增量和偏微分}
\[ f(x + \Delta x, y) - f(x, y) \approx f_x(x, y) \Delta x \]
左端是对x\uwave{偏增量}, 右端是对x\uwave{偏微分}
\paragraph{全增量}
$\Delta z = f (x + \Delta x, y + \Delta y) - f(x, y)$
\paragraph{可微与全微分}
\underline{全方向切线在同一平面}\\
若全增量$\Delta z$可表示为: $\Delta z = A \Delta x + B \Delta y + o(\rho)$
其中$A$, $B$仅与$x$, $y$有关, $\rho = \sqrt{(\Delta x)^2 + (\Delta y)^2}$,
那么称$z = f(x, y)$在$(x, y)$\uwave{可微分}\\
$dz = A \Delta x + B \Delta y$ 为$z = f(x, y)$在$(x, y)$的\uwave{全微分}
\paragraph{(全)可微与(偏)可导的关系}
\begin{mthm*}[可微一定可导]
如果$z = f(x, y)$在点$(x, y)$可微, 那么该函数在点$(x, y)$点偏导数$\frac{\partial z}{\partial x}$与$\frac{\partial z}{\partial x}$必定存在,
且全微分为
\begin{equation}
    dz = \frac{\partial z}{\partial x} dx + \frac{\partial z}{\partial x} dy \tag{全微分}
\end{equation}
又称\underline{叠加原理}
\end{mthm*}
\begin{mthm*}[可导不一定可微]
    函数$z = f(x, y)$的偏导数$\frac{\partial z}{\partial x}$, $\frac{\partial z}{\partial y}$在点$(x, y)$连续, 那么该函数在该点可微分
\end{mthm*}
\begin{figure}[ht]
    \centering
    \begin{tikzpicture}
        \node (A) at (0,0) {函数连续};
        \node (B) at (3,0) {可微分};
        \node (C) at (60:3) {函数可偏导};
        \node (D) at (6, 0) {连续偏导数};
        \draw[double distance=2pt] (A) -- (C);
        \draw[double distance=2pt, <-] (A) -- (B);
        \draw[double distance=2pt, <-] (C) -- (B);
        \draw[double distance=2pt, <-] (B) -- (D);
    \end{tikzpicture}
\end{figure}
\paragraph{全微分近似计算}
\section{多元复合函数求导法则}
\paragraph{通用复合} 对于$z = f(u, v)$, $u = \phi (x, y)$, $\Psi (x, y)$
\begin{equation}
\begin{pmatrix}
    \dfrac{\partial z}{\partial x} & \dfrac{\partial z}{\partial x}
\end{pmatrix}
=
\begin{pmatrix}
    \dfrac{\partial z}{\partial u} & \dfrac{\partial z}{\partial v}
\end{pmatrix}
\begin{pmatrix}
    \dfrac{\partial u}{\partial x} & \dfrac{\partial u}{\partial y} \\[8pt]
    \dfrac{\partial v}{\partial x} & \dfrac{\partial v}{\partial y}
\end{pmatrix}
\tag{多元复合函数通用求导法则}
\label{多元复合函数通用求导法则}
\end{equation}
\paragraph{全微分形式不变性质} 对于$z = f(u, v)$, $u = \phi (x, y)$, $v = \Psi (x, y)$, 且这两个函数具有连续偏导数
\begin{align*}
    dz &= \frac{\partial z}{\partial u} du + \frac{\partial z}{\partial v} dv \\
       &= \frac{\partial z}{\partial x} dx + \frac{\partial z}{\partial y} dy
\end{align*}

\begin{example}{设$z = e^u sin\ v$, $u = xy$, $v = x + y$, 求$\frac{\partial z}{\partial x}$, $\frac{\partial z}{\partial y}$}
    \begin{align*}
        dz &= d(e^u sin\ v) \\
           &= e^u sin\ v du + e^u cos\ v dv \\
           &= e^u sin\ v d(xy) + e^u cos\ v d(x + y) \\
           &= e^u sin\ v (ydx + xdy) + e^u cos\ v (dx + dy)
    \end{align*}
\end{example}

\section{隐函数求导法}
\paragraph{一个方程} 对于$F(X, y, z) = 0$若函数$F(x, y, z)$在点$P(x_0, y_0, z_0)$的某一领域内具有连续偏导数
\begin{equation}
    \frac{\partial y}{\partial x} = - \frac{F_x}{F_y} \tag{隐函数求导公式}
\end{equation}
\paragraph{方程组}\label{方程组定义隐函数求导}
对于$\begin{cases}F(x, y, u, v) = 0 \\ G(x, y, u, v) = 0\end{cases}$,
四个变量中一般只能有两个个变量独立变化,
因此可确定两个二元函数$\begin{cases}F(x, y, u(x, y), v(x, y)) = 0 \\ G(x, y, u(x, y), v(x, y)) = 0\end{cases}$ \\
应用\eqref{多元复合函数通用求导法则}则对两边对x求导可得
$\begin{cases}F_x  + F_u \frac{\partial u}{\partial x} + F_v \frac{\partial v}{\partial x} = 0 \\ G_x  + G_u \frac{\partial u}{\partial x} + G_v \frac{\partial v}{\partial x} = 0\end{cases}$\\
解得$
    \begin{cases}
        \frac{\partial u}{\partial x} = \frac{\left|\begin{smallmatrix}-F_x && F_v \\ -G_x && G_v\end{smallmatrix}\right|}{\left|\begin{smallmatrix}F_u && F_v \\ G_u && G_v\end{smallmatrix}\right|} = -\frac{1}{J}\frac{\partial(F, G)}{\partial(x, v)} \\
        \frac{\partial v}{\partial x} = \frac{\left|\begin{smallmatrix}F_u && -F_x \\ G_u && -G_x\end{smallmatrix}\right|}{\left|\begin{smallmatrix}F_u && F_v \\ G_u && G_v\end{smallmatrix}\right|} = -\frac{1}{J}\frac{\partial(F, G)}{\partial(u, x)}
    \end{cases}
$
\begin{define*}[雅可比行列式]
    \begin{equation}
    \frac{\partial (F, G)}{\partial (u, v)} = \begin{vmatrix}F_u & F_v \\ G_u & G_v\end{vmatrix} \tag{雅可比行列式} \label{雅可比行列式}
    \end{equation}
\end{define*}

\section{多元函数微分学的几何应用}
\subsection{一元向量值函数及其导数(导向量)}
\subsection{空间曲线的切线和法平面}\label{空间曲线的切线和法平面}
对于空间曲线$\Gamma = \begin{cases} x = \varphi (t) \\ y = \psi (t) \\ z = \omega (t) \end{cases}$切向量为
\[
    T = \begin{pmatrix} \varphi^{'} (t_0) && \psi^{'} (t_0) && \omega^{'} (t_0) \end{pmatrix}
\]
切线方程为
\[
    \frac{x - x_0}{\varphi^{'} (t_0)} = \frac{y - y_0}{\psi^{'} (t_0)} = \frac{z - z_0}{\omega^{'} (t_0)}
\]
法平面方程
\[
    {\varphi^{'} (t_0)}({x - x_0}) + {\psi^{'} (t_0)}({y - y_0}) + {\omega^{'} (t_0)}({z - z_0}) = 0
\]
\begin{example}
    {对于空间曲线$\begin{cases} F(x, y, z) = 0 \\ G(x, y, z) = 0 \end{cases}$求切线与法平面}
    将$\begin{cases} F(x, y, z) = 0 \\ G(x, y, z) = 0 \end{cases}$看作
    \[\begin{cases} F(x, y(x), z(x)) = 0 \\ G(x, y(x), z(x)) = 0 \end{cases}\]
    对两边求全导数得
    \[\begin{cases}F_x  + F_y \dfrac{\partial y}{\partial x} + F_z \dfrac{\partial z}{\partial x} = 0 \\[8pt] G_x  + G_y \dfrac{\partial y}{\partial x} + G_z \dfrac{\partial z}{\partial x} = 0\end{cases}\]
    应用\ref{方程组定义隐函数求导}可得到
    \[\begin{pmatrix}\dfrac{dx}{dx} && \dfrac{dy}{dx} && \dfrac{dz}{dx}\end{pmatrix} = \begin{pmatrix} 1 && -\dfrac{1}{J}\dfrac{\partial(F, G)}{\partial(x, z)} && -\dfrac{1}{J}\dfrac{\partial(F, G)}{\partial(y, x)} \end{pmatrix}, J = \frac{\partial(F, G)}{\partial(y, z)}\]
    乘J可得切向量
    \[T = \begin{pmatrix} \dfrac{\partial(F, G)}{\partial(y, z)} && \dfrac{\partial(F, G)}{\partial(z, x)} && \dfrac{\partial(F, G)}{\partial(x, y)} \end{pmatrix}\]
    由此可得切线方程和法平面
\end{example}
\subsection{曲面的切平面与法线}
对于隐式确定的曲面方程$F(x, y, z) = 0$法向量为
\[n = (F_x, F_y, F_z)\]
切平面与法线参考\ref{空间曲线的切线和法平面}

\section{方向导数与梯度}
\subsection{方向导数}
若$f(x, y)$在点$P_0(x_0, y_0)$可微分, 那么函数在该点沿任一方向$l$的方向导数存在
\[
    \left.\frac{\partial f}{\partial l}\right|_{(x_0, y_0)} = f_x\ cos\ \alpha + f_y\ cos\ \beta
\]
$cos\ \alpha$和$cos\ \beta$是方向$l$的方向余弦, $(cos\ \alpha, cos\ \beta)$为$l$的单位向量
\subsection{梯度}
对于$f(x, y)$在点$P_0(x_0, y_0)$的梯度为
\[grad\ f = \nabla f = f_x \vec{i} + f_y \vec{j}\]
如果$f$在该点可微分, 那么
\begin{align*}
    \left.\frac{\partial f}{\partial l}\right|_{(x_0, y_0)} &= f_x\ cos\ \alpha + f_y\ cos\ \beta \\
    &= grad\ f \cdot \vec{e_i} = |grad\ f|\ cos\ \theta
\end{align*}
其中$\theta = (\reallywidehat{grad\ f, e_i})$

\section{多元函数的极值及其求法}
\subsection{无条件极值}
若$z = f(x, y)$在点$(x_0, y_0)$的某领域内连续且有一阶及二阶连续偏导数, 又$f_x (x_0, y_0) = 0$, $f_y (x_0, y_0) = 0$,
令$f_{xy} (x_0, y_0) = A$, $f_{xy} (x_0, y_0)$, $f_{yy} (x_0, y_0) = C$
则取得极值条件如下:
\begin{itemize}
    \item $AC - B^2 > 0$时具有极值, 且当$A < 0$时有极大值, $A > 0$时有极小值
    \item $AC - B^2 < 0$时没有极值
    \item $AC - B^2 = 0$时不能确定
\end{itemize}
\subsection{条件极值 拉格朗日乘数法}
对于函数$z = f(x, y)$在附加条件$\varphi (x, y) = 0$下的可能极值点, 作拉格朗日函数
\[L(x, y) = f(x, y) + \lambda \varphi (x, y) = 0\]
求解方程组
\[
    \begin{cases}
        L_x = 0 \\
        L_y = 0 \\
        \varphi (x, y) = 0
    \end{cases}
\]
解得$(x, y)$即可能极值点
\end{document}
